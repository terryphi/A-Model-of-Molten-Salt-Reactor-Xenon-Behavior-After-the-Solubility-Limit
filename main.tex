\documentclass[twocolumn]{article}
\usepackage[utf8]{inputenc}
\newcommand{\XeP}{$^{135}$Xe }
\newcommand{\TeP}{$^{135}$Te }
\newcommand{\IP}{$^{135}$I }

\title{A Model of Molten Salt Reactor Xenon Behavior After the Solubility Limit}
\author{Terry Price}
\date{February 2019}

\begin{document}

\maketitle

\section{Introduction}
Nuclear reactors can be classified according to the their fuel's state of mater. Most reactors in operation today are solid fuel reactor. Example of solid fuel reactors include the Pressurized Water Reactor (PWR), Pressurized Heavy Water Reactor (PHWR), or Boiling Water Reactor (BWR).  There have been a number of nuclear reactors designed and developed which use a liquid fuel instead of a solid fuel.  One such liquid fuel reactor is the Molten Salt Reactor (MSR) which uses a molten alkali fluoride fuel salt as both the fuel matrix and primary working fluid. A reference text on MSRs, which includes contributions for researchers around the world, has been compiled by Dolan \cite{Dolan2017}. A description of the MSR's historical development and its upcoming involvement in  Generation IV is given by Serp et al. \cite{Serp2014}.

In an MSR with a thermal neutron spectrum, an actinium fluoride is dissolved in the fuel salt  circulates between a moderator and a heat exchanger.  The majority of power generated in a MSR is generated in the moderating region, circulates to a heat exchanger and is transferred to a secondary side through the heat exchanger.  

During the fission process,  \XeP

\bibliographystyle{unsrt}
\bibliography{bib}

\end{document}
